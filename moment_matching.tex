% Options for packages loaded elsewhere
\PassOptionsToPackage{unicode}{hyperref}
\PassOptionsToPackage{hyphens}{url}
%
\documentclass[
]{article}
\usepackage{xcolor}
\definecolor{material-grey}{RGB}{38,50,56}
\pagecolor{material-grey}
\color{white}
\usepackage{lmodern}
\usepackage{amssymb,amsmath}
\usepackage{ifxetex,ifluatex}


\begin{document}

\section{Query Release via Moment Matching}

In this section, we show how we can use moment-matching kernels as an efficient distance metric between distributions. We can choose a kernel such that the feature map contains all queries of interest, and the kernel trick allows us to  compute the worst expected difference between two distributions in polynomial time.

\subsection{Moment Matching Kernels}

\subsubsection{Reproducing Kernel Hilbert Spaces}

\textbf{Definition 1:} Let \(\mathcal{F}\) be a class of functions and
let \(p\) and \(q\) be the true data and fake data distribution
respectively, and \(X\) and \(Y\) be finite observations drawn iid from
\(p,q\) respectively. We then define the maximum mean discrepancy and
its empirical estimate as

\begin{align}
  MMD[\mathcal{F},p,q] &:= \sup_{f\in\mathcal{F}} ( \mathbb{E}_{x\sim p}[f(x)] - \mathbb{E}_{y\sim q}[f(y)])\\
  MMD[\mathcal{F},X,Y] &:= \sup_{f\in\mathcal{F}} ( \frac{1}{m} \sum_{i=1}^m f(x_i) -  \frac{1}N \sum_{i=1}^m f(y_i) )
\end{align}

We now let $\mathcal{F}$ be the unit ball in some reproducing kernel Hilbert space $\mathcal{H}$. By the representer theorem, there is a feature map $\phi(x)$ from $\mathcal{X}$ to $\mathbb{R}$ such that $f(x) = \langle f, \phi(x) \rangle_\mathcal{H}$. In canonical form, this feature map is $\phi(x) = k(x, \cdot)$. In this instance, we can express the MMD as the distance in $\mathcal{H}$ between mean embeddings \cite{GBR+12}.

\begin{equation}
  MMD^2[\mathcal{F}, p, q] = || \mathbb{E}_{x\sim p}[f(x)] - \mathbb{E}_{y\sim q}[f(y)] ||^2_\mathcal{H}
\end{equation}

\subsubsection{Query Classes as feature maps}

Using the kernel trick, we can efficiently find the function
$f \in \mathcal{F}$ s.t $||f||_\mathcal{H} \leq 1$ that maximizes the difference :

\[
  ||\mathbb{E}_{x\sim p}[f(x)] - \mathbb{E}_{y\sim q}[f(y)]||_\mathcal{H}
\]

\textbf{[TODO can't make the jump from supremum to sum]}

% \begin{align*}
% \mathcal{L}_{\mathrm{MMD}^2}&:= \|\frac{1}N \sum_{i=1}^N f (x_i)-\frac{1}{M} \sum_{j=1}^M f (y_j)\|^2 \\
% &=\frac{1}{N^2} \sum_{i=1}^N \sum_{i'=1}^N f (x_i)^{\top} f (x_{i'})-\frac{2}{N M} \sum_{i=1}^N \sum_{j=1}^M f (x_i)^{\top} f (y_j) +\frac{1}{M^2} \sum_{j=1}^M \sum_{j^N} use the kernel trick to rewrite this in terms of the kernel $k$ where $k(a,b) = \langle f(a), f(b) \rangle$.
% \end{align*}

% \textbf{}
% \begin{align*} 
%   \mathcal{L}_{MMD^2} &=\frac{1}{N^2} \sum_{i=1}^N \sum_{i'=1}^N k  (x_i, x_{i'}) - \frac{2}{N M} \sum_{i=1}^N \sum_{j=1}^M k (x_i, y_j) +\frac{1}{M^2} \sum_{j=1}^M \sum_{j'=1}^M k (y_j, y_{j'}) \end{align*}

\subsection{Boolean Kernels}



Note that maximizing this discrepancy is exactly equivalent to
maximizing the Wasserstein GAN objective. As such, if we can find a
kernel \(k\) in a RKHS such that its feature map \(\mathcal{F}\) 
describes a query class of interest, we can use
\(\mathcal{L}_{\mathrm{MMD}^2}\) as an efficient loss function for the
\emph{entire} class of queries.

One such kernel is the kernel corresponding to all monotone monomials of
length up to \(d\), which we denote by \(k_d\) \cite{PLA18}. {[}TODO
define \(\mathbb{B}\){]}

\[
  k_d (\mathbf{x}, \mathbf{x}'):= \langle f(\mathbf{x}), f (\mathbf{x}')\rangle_{K}=\sum_{\mathbf{i} \in \mathbb{B}^N_d} K_{\|\mathbf{i}\|}^{-1} \mathbf{x}^i \mathbf{x}^i
\]

Explicitly computing this would require summing
\(|\mathbb{B}_d^n| = O(n^d)\) terms. {[}TODO explain why, prove this{]},
Thus,

\[k (\mathbf{x}, \mathbf{x}') = \sum_{j=0}^d e^j \binom{\langle\mathbf{x}, \mathbf{x}'\rangle}{j}
=(1+b)^{ \langle \mathbf{x}, \mathbf{x}'\rangle}\]


where \(b\) determines the weight assigned to higher order polynomials.
a
{[}TODO all subsets kernel{]}

\textbf{NOte:} A supremum of convex functions is convex

\bibliography{works-cited}
\bibliographystyle{plain}


\end{document}
